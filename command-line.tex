\documentclass[aspectratio=169]{beamer}
\usetheme{smau}
\usecolortheme{smau}

%-------------------------------------------------------------------------------

%Information to be included in the title page:
\title{The Command Line}
\subtitle{KIPAC Computing Bootcamp 2023}
\author[S. Mau]{Sidney Mau}
\institute[Stanford]{Stanford University}
\date[\today]{\today}

%-------------------------------------------------------------------------------

\begin{document}

%-------------------------------------------------------------------------------

\frame{\titlepage}

\begin{frame}
	\frametitle{Outline}
	\tableofcontents
\end{frame}

%-------------------------------------------------------------------------------

\section{Interfacing with Files}

\frame{\sectionpage}

\begin{frame}
	\frametitle{The command line interface}
	The command line is a programmatic way to interface a computer through a \emph{shell}:
	\begin{itemize}
		\item Query the computer with some written command
		\item The command is run by the operating system
		\item Output is optionally printed back to the terminal
	\end{itemize}
\end{frame}

\begin{frame}
	\frametitle{Navigating directories and files}
	\begin{itemize}
		\item \texttt{pwd} -- print the present working directory
		\item \texttt{ls} -- print the contents of the present working directory
		\item \texttt{mkdir [dir]} -- make directories
		\item \texttt{cd [dir]} -- change directories
		\item \texttt{mv [source] [dest]} -- move a file from one location to another
		\item \texttt{cp [source] [dest]} -- copy a file in one location to another
		\item \texttt{rm [file]} -- delete a file
		\item \texttt{touch [file]} -- create an empty file
	\end{itemize}
\end{frame}

\begin{frame}
	\frametitle{Reading Files}
	\begin{itemize}
		\item \texttt{less [file]} -- interactively page through a file in the terminal
		\item \texttt{cat [file]} -- output all of a file
		\item \texttt{head [file]} -- output the first part of a file
		\item \texttt{tail [file]} -- output the last part of a file
	\end{itemize}
\end{frame}

\begin{frame}
	\frametitle{Some Quick Conventions}
	\begin{itemize}
		\item \texttt{q} -- close the current interactive interface (e.g., for \texttt{less}, \texttt{top})
		\item \texttt{[Tab]} -- autocomplete\footnote{The extent of autocomplete support can be very shell-dependent.}
	\end{itemize}
\end{frame}

\begin{frame}
	\frametitle{Writing Files}
	\begin{itemize}
		\item \texttt{nano [file]} -- something simple
		\item \texttt{vim [file]} -- something less simple
	\end{itemize}
\end{frame}

\begin{frame}
	\frametitle{Querying Files}
	\begin{itemize}
		\item \texttt{grep [patterns] [file]} -- find and output matches to patterns in a file
	\end{itemize}
\end{frame}

%-------------------------------------------------------------------------------

\section{Running Code}

\frame{\sectionpage}

\begin{frame}
	\frametitle{How do I run a program?}
	\begin{itemize}
		\item \texttt{python [script]} -- interpret and run a script with python
		\item \texttt{[command] \&} -- start a process in the background of the current shell
	\end{itemize}
\end{frame}

\begin{frame}
	\frametitle{Process Control}
	\begin{itemize}
		\item \texttt{ps} -- report current processes
		\item \texttt{jobs} -- show processes running in the current shell
		\item \texttt{fg} -- bring a process to the foreground
		\item \texttt{bg} -- resume a suspended process in the background (as if ran with \texttt{\&})
		\item \texttt{kill [pid]} -- kill a process
		\item \texttt{top} -- display current processes
	\end{itemize}
\end{frame}

\begin{frame}
	\frametitle{Signals}
	\begin{itemize}
		\item \texttt{[Ctrl]+[C]} -- interrupt current foreground process
		\item \texttt{[Ctrl]+[D]} -- signal end of current foreground process
		\item \texttt{[Ctrl]+[Z]} -- suspend current foreground process
	\end{itemize}
\end{frame}

%-------------------------------------------------------------------------------

\section{Wrapping Up}

\frame{\sectionpage}

\begin{frame}
	\frametitle{How does that work again\dots?}
	\begin{itemize}
		\item Ask for help!
		\item \texttt{man [item]} -- display the manual page for an item (e.g., \texttt{ls})\footnote{Not every command has manual pages.}
		\item \texttt{[command] --help} -- display help for a command\footnote{Not every command has a \texttt{--help} option.}
	\end{itemize}
\end{frame}

\begin{frame}
	\frametitle{Looking for more resources?}
	\begin{itemize}
		\item \url{https://swcarpentry.github.io/shell-novice/} -- \textit{The UNIX Shell} from Software Carpentry
		\item \url{https://missing.csail.mit.edu/} -- \textit{The Missing Semester of Your CS Education} from MIT
	\end{itemize}
\end{frame}

\begin{frame}
	\frametitle{Interested in more advanced topics?}
	\begin{itemize}
		\item Terminal multiplexers: \texttt{screen}, \texttt{tmux}
		\item Alternative shells: \texttt{zsh}, \texttt{fish}, \dots
		\item Alternative terminal emulators: \texttt{kitty}, \texttt{iTerm2}, \dots
	\end{itemize}
\end{frame}

\end{document}

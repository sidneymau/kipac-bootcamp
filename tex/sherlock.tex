\documentclass[aspectratio=169]{beamer}
\usetheme{smau}
\usecolortheme{smau}

\usepackage{multirow}
\usepackage{listings}
\usepackage{lstautogobble}
\usepackage{tikz}
\usetikzlibrary{tikzmark}
\usetikzlibrary{calc}
\usetikzlibrary{decorations.pathreplacing}
\usetikzmarklibrary{listings}

%-------------------------------------------------------------------------------

%Information to be included in the title page:
\title{Sherlock}
\subtitle{KIPAC Computing Bootcamp 2023}
\author[S. Mau]{Sidney Mau}
\institute[Stanford]{Stanford University}
\date[\today]{\today}

%-------------------------------------------------------------------------------

\begin{document}

%-------------------------------------------------------------------------------

\frame{\titlepage}

\begin{frame}
	\frametitle{Outline}
	\tableofcontents
\end{frame}

%-------------------------------------------------------------------------------

\section{Why use Sherlock?}

\frame{\sectionpage}

\begin{frame}
	\frametitle{What is Sherlock?}
	\begin{itemize}
		\item Sherlock is a research computing \emph{cluster} at Stanford
		\item KIPAC members have access to a number of proprietary resources at Sherlock through KIPAC and the School of Humanities and Sciences
		\item Sherlock is a great place to run intensive code, save datasets, collaborate with other KIPAC members, \dots
	\end{itemize}
\end{frame}

\begin{frame}
	\frametitle{Compute}
	\begin{itemize}
		\item My laptop: 8 cores, 32GB mem, 0 gpus
		\item Sherlock:
			\begin{center}
				\begin{tabular}{|l|l||r|r|r||r|r|r|}
					\hline
					\multicolumn{2}{|c||}{partition} & nodes & CPU cores & GPUs & \multicolumn{3}{|c|}{per-node} \\
					name   & public & total & total & total &   cores &   mem(GB) &  gpus \\
					\hline
					normal & yes    &   179 &  4820 &     0 &  20--32 &  128--256 &     0 \\
					bigmem & yes    &     9 &   504 &     0 & 24--128 & 384--4096 &     0 \\
					gpu    & yes    &    26 &   748 &   104 &  20--32 &  191--256 &     4 \\
					dev    & yes    &     4 &   104 &    64 &  20--32 &  128--256 & 0--32 \\
					\hline
					kipac  & no     &   112 &  3360 &     0 &  24--32 &  191--256 &     0 \\
					hns    & no     &   105 &  3728 &     0 & 20--128 & 128--1024 &     0 \\
					\hline
					owners & no     &  1471 & 47080 &   636 & 20--128 & 128--4096 &  0--8 \\
					\hline
				\end{tabular}
			\end{center}
	\end{itemize}
\end{frame}

\begin{frame}
	\frametitle{Storage}
	\begin{itemize}
		\item My laptop: 512GB
		\item Sherlock:
			\begin{itemize}
				\item \texttt{HOME}: 15GB
				\item \texttt{GROUP\_HOME}: 1TB
				\item \texttt{SCRATCH}: 100TB\footnote{Files not modified after 90 days are automatically deleted.}
				\item \texttt{GROUP\_SCRATCH}: 100TB\footnotemark[\value{footnote}]
				\item \texttt{OAK}: 1.3PB\footnote{Contact Phil Mansfield if you think you need access to Oak.}
			\end{itemize}
	\end{itemize}
\end{frame}

%-------------------------------------------------------------------------------

\section{Using Sherlock}

\frame{\sectionpage}

\begin{frame}[fragile]
	\frametitle{Connecting}
	\begin{itemize}
		\item Use \texttt{ssh} on the command line:
			\begin{verbatim}
				ssh <username>@login.sherlock.stanford.edu
			\end{verbatim}
			Note: this connects you to a login node; these are meant for basic file browsing, editing, etc.---do not run code on login nodes!
		\item Navigate a web browser to \url{login.sherlock.stanford.edu}
	\end{itemize}
\end{frame}

\begin{frame}
	\frametitle{Sherlock Commands}
	\begin{itemize}
		\item \texttt{sh\_quota} -- Print quotas for your filesystems
		\item \texttt{sh\_part} -- Print information about your partitions
		\item \texttt{sh\_dev} -- Start an interactive ``development'' session on a compute node
	\end{itemize}
\end{frame}

\begin{frame}
	\frametitle{Running Jobs}
	\begin{itemize}
		\item To run code, submit jobs using \emph{slurm}
		\item Slurm is a job management tool used by many high-performance computing clusters including Sherlock.
		\item To use slurm, you can write a script that requests resources and includes the steps to run your job.
		\item Use \texttt{sbatch} to submit your job to the \emph{queue}; slurm will dispatch jobs from the queue to the compute resources in a way that ensures resources are farily allocated among all users.
		\item \texttt{sbatch} scripts are shell scripts with special comments for job configuration
		\item \texttt{sbatch [script]} -- submit a slurm script
	\end{itemize}
\end{frame}

% sbatch options https://slurm.schedmd.com/sbatch.html

% TODO series of example scripts for different configs

% - https://docs.nersc.gov/systems/perlmutter/running-jobs/#example-scripts
% - https://www.sherlock.stanford.edu/docs/user-guide/running-jobs/#batch-jobs

\begin{frame}[fragile]
	\frametitle{Example sbatch script}
	\begin{lstlisting}[autogobble=true,basicstyle=\ttfamily,name=code]
		#!/bin/bash
		#SBATCH --partition=kipac
		#SBATCH --ntasks=1
		#SBATCH --ntasks-per-node=1
		#SBATCH --cpus-per-task=1
		#SBATCH --mem=4G
		#SBATCH --time=1:00:00
		
		source ~/setup.sh
		python script.py
	\end{lstlisting}
	\begin{tikzpicture}[overlay,remember picture]
		\draw[<-,overlay,blue] (pic cs:line-code-1-end) +(1em,.5ex) -- +(2.5,.5ex) node[right,thin] {in which shell the script is written};
		% \draw[<-,overlay,blue] (pic cs:line-code-2-end) +(1em,.7ex) -| (pic cs:line-code-7-end) +(1em,.7ex) -| +(2.5,1) node[above,thin] {arguments to \texttt{sbatch}};
		% \draw[decoration={brace,amplitude=0.5em},decorate,overlay,blue] (pic cs:line-code-2-end) -- (pic cs:line-code-7-end) node[right,thin] {arguments to \texttt{sbatch}};
		\draw[decoration={brace,amplitude=0.5em},decorate,overlay,blue]
			($([xshift=18em]pic cs:line-code-2-start)!([yshift=1ex]pic cs:line-code-2-start)!($([xshift=18em]pic cs:line-code-2-start)-(0,1)$)$) --
			($([xshift=18em]pic cs:line-code-7-start)!([yshift=0ex]pic cs:line-code-7-start)!($([xshift=18em]pic cs:line-code-7-start)-(0,1)$)$)
			node [right,pos=0.5,thin] {arguments to \texttt{sbatch}};
		\draw[decoration={brace,amplitude=0.5em},decorate,overlay,blue]
			($([xshift=12em]pic cs:line-code-9-start)!([yshift=1ex]pic cs:line-code-9-start)!($([xshift=12em]pic cs:line-code-2-start)-(0,1)$)$) --
			($([xshift=12em]pic cs:line-code-10-start)!([yshift=0ex]pic cs:line-code-10-start)!($([xshift=12em]pic cs:line-code-7-start)-(0,1)$)$)
			node [right,pos=0.5,thin] {code to be run};
		% \draw[decoration={brace,amplitude=0.5em},decorate,overlay,blue]
		% 	($(pic cs:line-code-2-end) +(1em,0)$)--
		% 	($(pic cs:line-code-7-end) +(1em,0)$)
		% 	node[right,thin] {arguments to \texttt{sbatch}};
		% \draw[<-,overlay,blue] (pic cs:line-code-8-end) +(1em,.7ex) -| (pic cs:line-code-9-end) +(1em,.7ex) -| +(2.5,1) node[above,thin] {code to be run};
	\end{tikzpicture}
\end{frame}

\begin{frame}[fragile]
	\frametitle{Example sbatch script}
	\begin{verbatim}
		#!/bin/bash
		#SBATCH --partition=kipac
		#SBATCH -N 1
		#SBATCH --ntasks-per-node=1
		#SBATCH -c 32
		#SBATCH --mem=0
		#SBATCH --time=1:00:00
		
		# content of job goes here
	\end{verbatim}
\end{frame}

\begin{frame}[fragile]
	\frametitle{Example sbatch script}
	\begin{verbatim}
		#!/bin/bash
		#SBATCH --partition=kipac
		#SBATCH --nodes=1
		#SBATCH --exclusive
		#SBATCH --mem=0
		#SBATCH --time=1:00:00
		
		# content of job goes here
	\end{verbatim}
\end{frame}

% tasks vs nodes vs cpus
% how to ask for a node:

%-------------------------------------------------------------------------------

\section{File Transfer}

\frame{\sectionpage}

\begin{frame}[fragile]
	\frametitle{Data Transfer Nodes}
	Sherlock has data transfer nodes specifically meant for moving larger amounts of data to/from Sherlock using \texttt{scp}, \texttt{rsync}, etc\ldots:
	\begin{verbatim}
		dtn.sherlock.stanford.edu
	\end{verbatim}
\end{frame}

\begin{frame}
	\frametitle{Globus}
	Sherlock has several Globus endpoints:
	\begin{itemize}
		\item \texttt{SRCC Sherlock}
		\item \texttt{SRCC Oak}
	\end{itemize}
\end{frame}

%-------------------------------------------------------------------------------

\end{document}
